\newpage
\fancyhead[C]{Thomas Turner}
\section{Return to Safety} \label{Return to Safety}


\subsection{Introduction and Philosophy}
Traditionally \gls{RTH} is executed in commercial drone products in case of detected faults and errors. This is usually sufficient and convenient for the user. However, this specific use case provides cause to expand this functionality. In cases of partial thrust failure, provided it is not above the critical value needed to maintain stability, the \gls{UAV} will have to use less aggressive control in order to cause no further degeneration and compensate for lower actuator saturation levels. This makes the \gls{UAV} less robust to disturbances and therefore introduces a significant probability of crashing. Similarly, if the battery state of charge is depleting faster than predicted and hence causing a state of charge failure you can model this as a probability of failure at any given point given the unpredictability of degenerating cells \cite{cell stuff}. Lastly, if adverse weather conditions are detected causing instability, similarly to if there is partial thrust failure, the device can no longer be assumed to be robust to the environmental disturbances and has a significant probability of crashing. Given the difficulty of retrieval, this means that traditional \gls{RTH} is not the optimal strategy in these specific cases and instead a \gls{RTS} strategy is more appropriate.

\paragraph{When to use \gls{RTH}}
\gls{RTH} is always preferable to to \gls{RTS} as if you trace back the path you came down and return to the start location you can ensure that the path is obstacle free and you have returned to a safe and convenient location. Therefore, \gls{RTH} should be used when the probability of failure at any specific instant on the path is negligible as the drone is robust to the environmental conditions, and when the drone has accurate location sensing for avoidance of large obstacles and obstructions. This means in cases of redundant sensor or communication faults, non safety critical sensor faults, or if commanded by the base station, \gls{RTH} should be executed.

\paragraph{Objectives}
The objective of the \gls{RTS} system is to provide a robust and flexible procedure that can be operated from the flight controller with minimal added memory and time complexity that provides optimal or near optimal routes to safe landing zones. It should also be easy to setup and operate by untrained ground operators.


\subsection{Cost Maps}
\subsubsection{Source data}
When considering \gls{RTS} it requires prior knowledge of the environment to be executed successfully. Therefore, this must be loaded into the drone using ground operators. The easiest and most efficient way to model this is using bird's eye images of the operating areas and marking the obstructions, suspected mined regions, mildly dangerous regions, and safe landing regions, with corresponding \gls{GNSS} co-ordinates.
\paragraph{Satellite Images}
Where satellite images are available and accurate they are the easiest and most efficient option. However, given the nature of environments effected by war, there may have been significant changes to the local environment since the last satellite image. Furthermore, seasonal changes can make satellite images less effective as trees may be less visible from winter images due to a lack of leaves.
\paragraph{Surveillance Images}
Standard consumer level drones can provide images with tagged \gls{GNSS} co ordinates that can be used, however, this requires extra hardware and time to execute. Therefore, this should be avoided where satellite images are sufficient.

\subsubsection{Graphical User Interface}
\paragraph{Platform} 
The \gls{GUI} was considered from the start to maximise usability by untrained ground operators. I built the \gls{GUI} as a simple python script that takes an image path and uses hardware-independent click locations and keystrokes to operate. It then gives he output as a .txt file with a standard format. This means that the process can be run on any local hardware available.
\paragraph{User interactions}
It is a simple program where the user can click on a region to set it a colour representing the classification. A key development was using semi-transparent colours so the base image can be seen through the shade of classification colour allows for a more natural filling experience. Furthermore, some utility functions including multi-square filling, clicking on already classified regions to deselect and auto-filling regions were added to make the process easier and faster. The process from satellite image (from Google Earth), to cost map, to a flow map is shown in \ref{fig:cost map net}. The only information uploaded to the drone is \ref{fig:cost map flow} where the arrows denote where to go next and the circles denote where, if there, the \gls{UAV} should land.

\subsubsection{Tessellated surfacing}
\paragraph{Why hexagons} While squares and triangles are more widely used, they are worse than hexagons for this application where the path planning is between central nodes and the ground operator intuitiveness must be considered. This is because, hexagons exhibit uniform adjacency meaning the distance between the central point of each hexagon and its neighbours is equal, unlike triangles and rectangles. Differences in distances are difficult to interpret as the ground operator as they change the probability of survival between nodes of the same class depending on orientation. Furthermore, travelling node to node on the diagonals of squares goes through a point of four intersection where the classification is undefined. Therefore if the user added two obstacles diagonally connected it is unknown if you could travel diagonally between them, this ambiguity creates issues that you do not face when using hexagons as while at the intersections the classifications are still undefined, when travelling centre to centre you never cross an intersection of more than two hexagons.
\paragraph{Mapping hexagons to \gls{GNSS}} \label{para:Mapping hexagons}
The two key methods of mapping from hexagons to Cartesians either uses two axial co-ordinates or using row and column values with an offset described in \cite{MappingHexagons}. Since the later is simpler to translate into matrix form it was used however, both are easy to implement. Once the converted into Cartesian form you multiply the values by the scaling factor to get offset from the origin. This is combined with the known location of the origin to generate the \gls{GNSS} tags for the centre of each hexagon.

\subsubsection{Efficient filling}
\paragraph{Auto-filling from Path Planning}
Having the ground user filling in both the path planning polygonal zones and then refilling the cost map is inefficient as there is a lot of information that is shared between the two. Therefore, provided the images used are the same, the cost map automatically classifies hexagons with an obstacle listed within its bounds as an obstacle region. It also automatically classifies the region of interest as dangerous to land as it is a mined area. This reduces the number of zones required to be filled by the end user significantly. 
\paragraph{Dynamic Zooming}
In developed or otherwise difficult operating environments higher resolutions may be needed, however, often the major zones will remain the same. Therefore, the user will have two different views, the major view that is used to fill in large zones and the minor view that the user accesses in order to fill in smaller more detailed zones. The entire map is in fact the smaller zones but this abstraction means it is easy and quick to fill in detailed maps.
\paragraph{Machine Learning Methods}
The cost map generating program not only records the final results but all of the ground operator times and clicks. This output file is  sent to the development team if the ground user has not disabled this functionality due to safety reasons. This will provide data in order to create new tools based on tasks that are taking more time to complete. It will also allow development of predictive models that can automatically fill in zones in a seamless manor using the source image and the user clicks as input data. For example, a current bottleneck is linear obstacle lines for example power, hedge or tree lines, this is tedious to fill in. However, with a machine learning approach it would mark what it suspects is the line in a lighter black after the first user click in the region and then the user can press Tab to fill in the prediction or ESC to remove the prediction. Another key improvement would be automatic region filling where, for example, the user has filled in half a field, a good prediction may be that the other half would follow the same classification and would similarly be suggested as above to the user.

\begin{figure}[htbp]
  \centering
  \begin{subfigure}[b]{0.32\textwidth}
    \includegraphics[width=\textwidth]{figs/Thomas/Return To Safety/50.19.27.N 36.55.32.E.png}
    \caption{Satellite Image}
    \label{fig:cost map satellite}
  \end{subfigure}
  \hfill
  \begin{subfigure}[b]{0.32\textwidth}
    \includegraphics[width=\textwidth]{figs/Thomas/Return To Safety/50.19.27.N 36.55.32.E Cost.png}
    \caption{Cost Map}
    \label{fig:cost map cost}
  \end{subfigure}
  \hfill
  \begin{subfigure}[b]{0.32\textwidth}
    \includegraphics[width=\textwidth]{figs/Thomas/Return To Safety/50.19.27.N 36.55.32.E Flow.png}
    \caption{Flow Map}
    \label{fig:cost map flow}
  \end{subfigure}
  
  \caption{Geospatial analysis at coordinates 50.19.27.N 36.55.32.E}
  \label{fig:cost map net}
\end{figure}
\subsection{Fault Detection, Quantification and Control}

\subsubsection{State of Charge}
\paragraph{Component Ageing}
Cell ageing and actuator ageing, if not properly accounted for, can cause inaccurate state of charge predictions. Therefore, the planned path may exceed the limit of the \gls{UAV} leading to a state of charge fault and the necessity of \gls{RTH} if the state of charge is within 15\% of the predicted required state of charge required for \gls{RTH}.
\paragraph{Cell Failure}
\gls{LiON} batteries can have a thermal runaway \cite{REF}. This is seen with a spike in temperature from the \gls{BMS} telemetry data and should lead to immediate landing as it is an extreme risk of fire and further damage \cite{REF} that makes \gls{RTS} not feasible.

\subsubsection{Actuator Fault}
\paragraph{Causes and Modelling}
A loss of actuator effectiveness can be either from motor damage or propeller damage. The possible causes of motor damage specific to the operating environments include: ... . For propeller damage this could be from impacts, thermal effects or fatigue cracks. In either case the actuator can be modelled as a reduced thrust/watt effect.
\paragraph{Control Strategy}
The simplest solution is to tune the output signals from the controller up until the thrust of the actuator matches the expected value, this allows the control loop to be unchanged. However, this abstracts from the physical effects. Firstly, the actuator will saturate at a lower thrust value meaning the control loop will not perform as expected and can lead to failure in typically non-failure states. Furthermore, by increasing the output on the actuator it will likely cause the current defect to degenerate as the loads increase on the damaged actuator. Therefore, a \textbf{Gain Scheduling} approach is used with a selection of less aggressive controllers to match different tuning magnitudes. The relationship being that as the tuning magnitude increases, the controller should be less aggressive so that the peak outputs reduce. However, this means that the controller cannot tolerate disturbances of the same magnitude. There should be varying controller values for different tuning magnitudes preloaded onto the drone and the drone should update which one is being used real-time by using the estimated value of tuning magnitude. Other approaches such as Model Reference Adaptive Control can provide more optimal solutions however, they come at the cost of increased computing complexity and less predictable responses.

\subsection{Emergency Path Planning}

\subsubsection{Objective and Cost Function}
\paragraph{Objective}
The key objective is to reduce the downtime and cost of operation of the drone. This means both the cost of damage to the drone through crashing and the difficulty of retrieval need consideration.
\paragraph{Cost Considerations}
For this specific usage case, path planning has to consider the probability of failure at all instances along the path. 
\paragraph{Cost Function}
Modelling the probability of surviving a node to node traversal between adjacent nodes as constant, $p$, the cost function of any path of length $D$ is shown in \ref{eq:cost function}.
\begin{equation}\label{eq:cost function}
    E(x) 
    = \sum_{n=1}^{D-1} c_{crashing}\bigl(x_n\bigr)\, (1-p) \,p^n 
    \;+\; c_{landing}\bigl(x_D\bigr)\, p^D
\end{equation}

\subsubsection{Weather induced failure}
\paragraph{Cause}
When a gust of wind exceeds the control capabilities of the drone it will cause an extreme disturbance and likely failure resulting in a crash. This is especially important to consider when their is an actuator fault due to the less aggressive control strategies deployed meaning that the maximum rejected gust becomes lower in magnitude and therefore more likely to occur. 
\paragraph{Gust Modelling}
Using the \textbf{Meteomatics API} forecast and modelling in \ref{SAM GRACE} the gustiness of the operating environment can be approximated. From this a model, the probability that a gust exceeds the known maximum rejection gust speed for the current control gains per second, $\lambda$, is calculated as shown in \ref{eq:p calc}. The specifics of the calculation of $\lambda$ are an area for future work, however, they would also greatly benefit from the collection of real-time data for the drone in testing.

\begin{equation}\label{eq:p calc}
    p = (1-\lambda)^{\frac{nodal\_distance}{speed}}
\end{equation}

\subsubsection{Search Algorithms}
\begin{algorithm}
\caption{Exhaustive Search}
\label{alg:search}
\begin{algorithmic}[1]
    \For{each Node $n$}
        \State $\textit{min\_cost} \gets \infty$
        \State $n.path\_cost$ = \Call{search}{$n$, $p$, $1$, $0$, $depth$, $\textit{min\_cost}$}
    \EndFor
    
    \Function{search}{$n$, $p$, $p\_alive$, $cost$, $depth$, \textbf{ref} $\textit{min\_cost}$}
        \State $\textit{current\_cost} \gets cost + p\_alive \cdot n.\textit{landing\_cost}$
        \If{$\textit{current\_cost} < \textit{min\_cost}$}
            \State $\textit{min\_cost} \gets \textit{current\_cost}$
        \EndIf
        \If{$depth = 0$ $\mathbf{or}$ $n.\textit{landing\_cost} = 0$ $\mathbf{or}$ $cost > \textit{min\_cost}$}
            \State \Return
        \EndIf
        \For{each $\textit{neighbour}$ in $n.\textit{neighbours}$}
            \State $p\_alive\_next \gets p\_alive \cdot p$
            \State $cost\_next \gets cost + p\_alive \cdot (1 - p) \cdot n.\textit{crashing\_cost}$
            \State \Call{search}{$\textit{neighbour}$, $p$, $p\_alive\_next$, $cost\_next$, $depth - 1$, $\textit{min\_cost}$}
        \EndFor
    \EndFunction
\end{algorithmic}
\end{algorithm}
\begin{algorithm}
    \caption{Smooth}
    \label{alg:flow}
\begin{algorithmic}[1]
    \While{not $Converged$}
        \State $Converged \gets TRUE$
        \For{each Node $n$}
            \For{each \textit{neighbour} in $n.\textit{neighbours}$}
                \State $\textit{movement\_cost} \gets n.\textit{crashing\_cost} \cdot \frac{1 - p}{2} +  \textit{neighbour}.\textit{crashing\_cost} \cdot \frac{p(1 - p)}{2}$
                \State $\textit{total\_cost} \gets \textit{movement\_cost} + p \cdot \textit{neighbour.path\_cost}$
                \If{$\textit{total\_cost} < n.\textit{path\_cost}$}
                    \State $n.\textit{path\_cost} \gets \textit{total\_cost}$
                    \State $Converged \gets FALSE$
                \EndIf
            \EndFor
        \EndFor
    \EndWhile
\end{algorithmic}
\end{algorithm}
\paragraph{Exhaustive Search}
The simplest and easiest to understand algorithm is a simple search algorithm as shown in \ref{alg:search}. This recursively calls search until it searches all viable lines. There are steps taken to increase the efficiency, including pruning lines that already exceed the minimum cost as the cost is monotonic increasing and automatically terminating lines when they are above the best terminal state possible. However, the worst case time complexity remains $O(|V|6^{depth})$ which means you cannot get guaranteed results at high depths quickly. The key use case issue is that when \textit{p} is close to 1, optimal paths can be very long and exceed the depth and therefore cause lower accuracies.
\paragraph{Smoothing}
To combat the issues with \ref{alg:search}, \ref{alg:flow} was developed. The key idea behind this algorithm is that it smooths out the differences in path cost between neighbouring nodes. This means you get a guaranteed correct answers when the nodes have fully converged. This algorithm is a Bellman-Ford style algorithm and therefore has a time complexity of $O(|E|\times |V|)$ \cite{REF}. In dense graphs this becomes  $O(|V|^3)$ however, we know $|E| \leq |V| \times 6$ as we are using hexagons that can have a maximum of 6 neighbours, therefore the actual complexity is $O(|V|^2)$. 

\paragraph{Optimal Algorithm}
Matching the complexities and assuming the computation time for each is similar per operation we can derive a complexity matched depth of \ref{eq:complexities}
\begin{align}\label{eq:complexities}
    6\times|V|^2 &\approx |V|\times6^{\textit{depth}} \nonumber \\
    |V| &\approx 6^{\textit{depth} - 1} \label{eq:vertex_depth} \\
    \textit{depth} &\approx \log_6(|V|) + 1 \nonumber
\end{align}
This means that if the below the depth of matching, if the performances are equivalent the search method should be used, if the performance is lower it should not be used.
\paragraph{Testing}
Using 10 random satellite images from the Kharkiv region, 10 cost maps were generated including visible obstacles and assumed regions of interest. Then starting at each node of the region of interest 20 seperate trials are run using the search or the smoothing algorithm and their performances are calculated using random failures in line with \textit{p}.

\subsubsection{Real-time application}
\paragraph{Pre-Compute}
Carrying out memory and time complex operations on real-time hardware should always be avoided. This is because with real time systems, loops require specific timings for optimal use. Therefore, if we have algorithms that do not operate to precise timings the system can fail or be forced to be less efficient. If all the values are pre-computed and uploaded to the drone with the optimal path the time and memory complexity becomes $O(1)$ which supports precise, timed loops. 
\paragraph{$p$ ranges}
While the value of $p$ can take any value between 0 and 1, there can be only 7 outputs for the next best move (going to each of the 6 neighbours or landing). Therefore, instead of creating a specific map for a specific value of $p$ on the Flight Controller you can pre-compute all the ranges of $p$ that would cause each of these outcomes and get the exact same result for the only thing that matters to the drone, which action should be taken next.
\paragraph{Deployment}
For a deployed algorithm consistent timings and a minimal memory footprint are essential. This allows for the path planning loop to run at higher speeds as with lower memory requirements faster memory access forms are available and with consistent timings you can maintain consistency in looping speed. To reduce the memory intensity, instead of storing ranges of values you store a sorted array of values going from the smallest $p$ to the greatest. The first time the specific value of $p$ crosses the threshold $p$, you record the required action and if no value crosses the threshold then you default to landing. Finally, to convert row and column indexes instead of the nodes into \gls{GNSS} co-ordinates required for headings you can use a precomputed reference map to ensure reliable access times, or work it out using the scaling factor and a reference co-ordinate to reduce the memory requirements.
\paragraph{Memory Management}
The memory on the \gls{MCU}s consists of Flash and \gls{SRAM}. Flash is static memory that has to be pre-loaded onto the board before each run whereas \gls{SRAM} is volatile but has faster access times. Therefore, the optimal strategy is to load the waypoints, maps and gains from the preset values in Flash before a flight into \gls{SRAM}. Furthermore, the Flight Controllers \gls{MCU} contains 2 megabytes of Flash, 64 kilobytes of \gls{ITCM} configured \gls{RAM} 128 kilobytes of \gls{DTCM} configured \gls{RAM} \cite{REF}. These allow for precise timings and should therefore be used for looped processes including control loops and heading calculations. Whereas, for less timing critical actions or varying time operations such as actuator tests the regular \gls{SRAM} can be used. However, for the \gls{GNSS} module \gls{MCU} there is just 128 kilobytes of Flash and 32 kilobytes of \gls{SRAM}.
\paragraph{Memory Requirement Reduction}
For maps 256 nodes or smaller instead of using 2 byte floats and 4 byte pointers we use a single array of reduced nodes as shown in \ref{lst:reducednode}. Slightly less precise $p$ values should not cause any major issues and instead of pointers the nodes are accessed with indexes. This is shown in \ref{lst:reducednode} and is 14 bytes per node compared to in \ref{lst:node} which is 40 bytes per node representing 35\% of the initial size.  This smaller size means that for low resolution maps you can get ultra-low memory requirements. A viable strategy may be to have two maps representing the higher and lower zoom levels of the cost map where the lower resolution map (where each node is equal to the worst node within it) is loaded onto the \gls{GNSS} module and the higher resolution map is loaded into the more powerful Flight Controller. For example, for a 1536 node high resolution map it requires 61.44 kilobytes of memory which is possible on the Flight Controller but when reduced to a 256 node low resolution maps it requires 3.584 kilobytes of memory, 5.83\% of the original size. This comes at the cost of lower performance but opens up \gls{RTS} even if the flight controller is not functional. The strategy of setting the low resolution node to the worst value in the high resolution space covered means a more conservative approach in line with the less precise control loop executed away from the Flight Controller. Further non-guaranteed optimal methods are available for further memory reductions,  while there are 7 options available for the next move at each node, usually it is just a threshold value at which you should land and then one viable move after a threshold $p$ value. This node is shown in \ref{lst:ultrareducednode} and contains just 4 bytes, therefore requiring only 1.024 kilobytes for a 256 node map. This can be used in cases of extremely low memory devices.
\begin{lstlisting}[caption={Node Structure},label={lst:node}]
struct Node {
    uint16_t row, col;      // Coordinates
    float16 p_values[6];    // Probability thresholds
    Node* actions[6];       // Connected nodes
};
\end{lstlisting}

\begin{lstlisting}[caption={Reduced Node Structure},label={lst:reducednode}]
struct Node {
    uint8_t row, col;       // Coordinates
    uint8_t p_values[6];    // Probability thresholds
    uint8_t actions[6];     // Connected nodes
};
\end{lstlisting}

\begin{lstlisting}[caption={Ultra Reduced Node Structure},label={lst:ultrareducednode}]
struct Node {
    uint8_t row, col;       // Coordinates
    uint8_t p_thresh;       // Probability threshold
    uint8_t next;           // Where to go next
};
\end{lstlisting}

\textbf{NEED GRAPH of 3 different memories}
\textbf{NEED GRAPH of PSEUDO CODE}


\subsection{Hardware Deployment}

\subsubsection{Real-time application}
\paragraph{Pre-Compute}
Carrying out memory and time complex operations on real-time hardware should always be avoided. This is because the flight controller is a single core computer and ...
\paragraph{Number of maps required}
Show how the expected cost decreases and then stagnates

\subsubsection{Modelling weather induced failure}
\paragraph{Weather Forecasting}
Use OpenMeteo
\paragraph{Operator Input}
Use fuzzy logic
\paragraph{Real-Time Influence}
posterior updates using \gls{IMU data}

\subsubsection{Improvements with use}
\paragraph{Physical analysis}
\paragraph{Telemetry analysis}
\paragraph{Data collection and prediction}
 