\subsection{Fault Detection, Quantification and Control}

\subsubsection{State of Charge}
\paragraph{Component Ageing}
Cell ageing and actuator ageing, if not properly accounted for, can cause inaccurate state of charge predictions. Therefore, the planned path may exceed the limit of the \gls{UAV} leading to a state of charge fault and the necessity of \gls{RTH} if the state of charge is within 15\% of the predicted required state of charge required for \gls{RTH}.
\paragraph{Cell Failure}
\gls{LiON} batteries can have a thermal runaway \cite{REF}. This is seen with a spike in temperature from the \gls{BMS} telemetry data and should lead to immediate landing as it is an extreme risk of fire and further damage \cite{REF} that makes \gls{RTS} not feasible.

\subsubsection{Actuator Fault}
\paragraph{Causes and Modelling}
A loss of actuator effectiveness can be either from motor damage or propeller damage. The possible causes of motor damage specific to the operating environments include: ... . For propeller damage this could be from impacts, thermal effects or fatigue cracks. In either case the actuator can be modelled as a reduced thrust/watt effect.
\paragraph{Control Strategy}
The simplest solution is to tune the output signals from the controller up until the thrust of the actuator matches the expected value, this allows the control loop to be unchanged. However, this abstracts from the physical effects. Firstly, the actuator will saturate at a lower thrust value meaning the control loop will not perform as expected and can lead to failure in typically non-failure states. Furthermore, by increasing the output on the actuator it will likely cause the current defect to degenerate as the loads increase on the damaged actuator. Therefore, a \textbf{Gain Scheduling} approach is used with a selection of less aggressive controllers to match different tuning magnitudes. The relationship being that as the tuning magnitude increases, the controller should be less aggressive so that the peak outputs reduce. However, this means that the controller cannot tolerate disturbances of the same magnitude. There should be varying controller values for different tuning magnitudes preloaded onto the drone and the drone should update which one is being used real-time by using the estimated value of tuning magnitude. Other approaches such as Model Reference Adaptive Control can provide more optimal solutions however, they come at the cost of increased computing complexity and less predictable responses.
