\newpage
\fancyhead[C]{Thomas Turner}
\section{Custom Hardware} \label{section:Custom Hardware}
\subsection{Introduction and Philosophy}
Custom hardware is a key novelty to our design and helps us reduce our cost and increase our performance for the specific task at hand. This is because, our custom hardware affordably mitigates the specific risks of operation over the minefield. The key risk not faced by similar commercial projects is the fact we cannot safely carry out emergency landing procedures when a fault is detected, or in fact crash when a fault is incurred due to the difficulty of retrieval. Not only does it possibly damage the imaging sensors onboard but it comes at the significant opportunity cost of not clearing the mines if the drone is lost. 
\paragraph{When to use custom components}
\gls{COTS} options should always be considered before any custom part as if there is something that already meets the design requirements it is often the better choice. This is because you do not need to invest the design and development time into making a new device and the current manufacturer benefits from greater economies of scale. However, often when choosing \gls{COTS} options they can cause compromises to cost and specification that make custom hardware appropriate in specialist applications such as this project.
\paragraph{Objectives}
All hardware should be easy and quick to debug even by unskilled ground operators, it should support redundancy in both sensors and in communications. They should also line up with our sustainability objectives by ensuring all components are \gls{RoHS} compliant and our modularity objectives by having clear and easy interfaces.


\subsection{Design Considerations}

\subsubsection{Trace Lengths}
\paragraph{Signal Integrity and Timing}
All traces have propagation delay that increases linearly with length. This means that the longer the trace, the longer the delay. Furthermore, the longer the length, the greater effect impurities and crosstalk have creating a nosier signal. Therefore, traces should be kept as short as possible.
\paragraph{Length Matching}
In synchronous communication methods the clock signal must be in sync with the data transmission. This means that the lines should be as similar in length as possible so that they arrive at the same delay. The faster the communication rate the stricter this must become as the clock signal frequency is increased. Therefore, signal speeds should be as low as possible to increase robustness.
\paragraph{Effect of vias}
Vias allow for the transference of signals between planes which is often necessary for routing. However, they introduce extra impedance, noise and delay \cite{source} and therefore should be avoided. Furthermore, for synchronous high speed signals they should be included in trace length considerations as the line with fewer vias should be tuned to have a longer length to achieve the same delay. For this application, as none of the signals are greater than \textbf{highest speed} tuning exact timings is not necessary.

\subsubsection{Touch Protection}
\paragraph{\gls{ESD}}
\gls{ESD} occurs when there is a significant static charge built up in a ground operator or contacting surface that is then shorted in contact with the board. As electro-static voltages can be \cite{REF} this can cause significant damage to devices.
\paragraph{Protection Devices}
In order to protect against \gls{ESD} the circuit must provide a low resistance pathway to the ground. This can be using \gls{TVS} diodes that above certain voltages act like a wire that diverts the current flow. They can also be mitigated using voltage dependent resistors that increase resistance with voltage and absorb the errors. Lastly, by using low resistance resistors in series it decreases the current spike. In my designs I make use of both \gls{TVS} diodes and series resistors due to their simplicity and effectiveness.
\paragraph{Component Placement}
\gls{ESD} is largely from interactions with people, therefore, efforts to mitigate the effects should be as close to these expected points as possible. This is because the current loop to the ground should be as small as possible in order to minimise the damage to components. Therefore, in my designs \gls{TVS} diodes were placed near all interfaces and removable memory devices as these are the most likely points of contact.

\subsubsection{Component Placement}
\paragraph{Utility Regions}
Components with similar functionalities should be kept separate. For example, all digital signals should be kept away from analogue signals as the high frequency digital signals can induce noise in the analogue signals. In my design there are only digital signals and therefore the key areas are between the power management systems, sensors and communications that are all on separate board areas.
\paragraph{Thermal Considerations}
Heat dissipating elements, typically diodes and resistors, can cause damage to electronic components if not sufficiently managed. Therefore, heat sinks or controlled airflows are required if there are high power draws. This consideration is why on my devices Buck Converters of above 90\% \cite{REF} are used instead of the less efficient \gls{LDO} which would have an efficiency of 66\% when converting from 5V to 3.3V. This, in addition to the low power draws of all other components, means that no explicit thermal management is needed.

\subsubsection{Trace Widths and Spacing}
\paragraph{Thermal Management}
\paragraph{Impedance Control}
\paragraph{Crosstalk}

\subsubsection{Layers}
\paragraph{Two Layers}
\paragraph{Four Layers}
\paragraph{Six or more Layers}

\subsubsection{Component Selection}
\paragraph{Procurement Cost}
\paragraph{Mounting}
\paragraph{Footprint and Volume}
\subsection{Custom Components}

\subsubsection{Component Selection}
\begin{table}[htbp]
  \centering
  \begin{tabular}{lcccc}
    \hline
    \textbf{Name} & \textbf{Price/100} & \textbf{Gyro Noise} & \textbf{Accel Noise} & \textbf{Gyro Freq.} \\
    & & (mdps/$\sqrt{\text{Hz}}$) & (µg/$\sqrt{\text{Hz}}$) & (kHz) \\
    \hline
    ICM-42688-P & 329 & 2.8 & 70 & 32 \\
    MPU-6500    & 423 & 10  & 300 & 32 \\
    BMI323      & 157 & 7   & 180 & 12.5 \\
    \hline
  \end{tabular}
  \caption{Comparison of Selected IMUs}
  \label{tab:imu-comparison}
\end{table}
\footnote{\url{https://invensense.tdk.com/products/motion-tracking/6-axis/icm-42688-p/}}
\footnote{\url{https://invensense.tdk.com/products/motion-tracking/6-axis/mpu-6500/}}
\footnote{\url{https://www.bosch-sensortec.com/products/motion-sensors/imus/bmi323}}


\begin{table}[htbp]
  \centering
  \begin{tabular}{lrrrrrr}
    \toprule
    \textbf{Name} & \textbf{Flash (KB)} & \textbf{RAM (KB)} & \textbf{Clock (MHz)} & \textbf{Price} & \textbf{CAN} & \textbf{Pins} \\
    \midrule
    MSPM0G3506SRHBR & 32 & 32 & 80 & 87 & 1 & 32 \\
    STM32H743VGT6 & 1000 & 1000 & 480 & 737.21 & 2 & 100 \\
    STM32F765VGT7 & 1000 & 512 & 216 & 812 & 3 & 100 \\
    \bottomrule
  \end{tabular}
   \caption{Comparison of MCUs}
  \label{tab:mcu-comparison}
\end{table}
\footnote{\url{https://www.ti.com/product/MSPM0G3506/part-details/MSPM0G3506SRHBR}}
\footnote{\url{https://www.st.com/en/microcontrollers-microprocessors/stm32h743vg.html}}
\footnote{\url{https://www.st.com/en/microcontrollers-microprocessors/stm32f765vg.html}}

\begin{table}[htbp]
  \centering
  \begin{tabular}{lccccccc}
    \toprule
    \textbf{Name} & \textbf{Price/100} & \textbf{RTK} & \textbf{Heading Accuracy (deg)} & \textbf{Velocity Accuracy (m/s)} & \textbf{Update Rate (Hz)} & \textbf{Dead Reckoning} & \textbf{Precision (m)} \\
    \midrule
    MAX-F10S-00B & 1109 & No  & 0.3 & 0.05 & 10 & No  & 1    \\
    ZED-F9P-05B & 9082 & Yes & 0.3 & 0.05 & 5  & No  & 0.01 \\
    LC29HEAMD & 4091 & Yes & N/A & 0.03 & 10 & No  & 0.01 \\
    LC29HAAMD & 769  & No  & N/A & 0.1  & 10 & No  & 1    \\
    LG69T-AP & 5877 & Yes & N/A & 0.1  & 10 & Yes & 0.01 \\
    \bottomrule
  \end{tabular}
   \caption{Comparison of GNSS Modules}
  \label{tab:gnss-modules}
\end{table}
\footnote{\url{https://www.u-blox.com/en/product/max-f10s-module}}
\footnote{\url{https://www.u-blox.com/en/product/zed-f9p-module}}
\footnote{\url{https://www.quectel.com/product/gnss-lc29h/}}
\footnote{\url{https://www.quectel.com/product/gnss-lg69t-series/}}


\paragraph{\gls{IMU}}
Price and noise are the key considerations. While the BMI323 is cheaper, and should be fine for a 8kHz control system, it does not have the specifications required for further modification, improvement and advanced sensor fusion. This may be needed if imaging is required above 10 Hz and therefore needs specific dead reckoning between \gls{GNSS} measurements. Therefore, the ICM-42688-P was selected for use on all boards. It is used even on the \gls{GNSS} module even though the BMI323 could be used as the lower specification option so that the control characteristics are as similar as possible to the flight controller to reduce development and testing requirements.
\paragraph{\gls{MCU}}
For the more complex flight controller the STM32H743VGT6 is teh best option given its \gls{RAM}, cost and support of dual \gls{CAN} buses. However, for the less computationally intense operations required for the \gls{GNSS} module the smaller footprint (due to the lower number of pins) and lower priced MSPM0G3506SRHBR is a better option.
\paragraph{\gls{GNSS} Module}
\ref{tab:gnss-modules} shows the options considered for the \gls{GNSS} module. Exact positions for where the images were taken is required, this is possible using a \gls{RTK} setup with a base station sending out correction vectors over LoRA \cite{a source}, taking the accuracy to within 1cm at 10 Hz. \footnote{\url{the data sheet}}. Furthermore, in the case of the loss of all \gls{GNSS} signal built in dead reckoning, allowing \gls{RTS}, is also a desirable feature. Therefore, the LG69T-AP is selected. However, this is significantly more expensive than the \gls{GNSS} modules without these features, therefore, for the backup system the MAX-F10S-00B was selected. This is because, the improved velocity and heading functionalities compared to the LC29HAAMD. This is possible because, in the case of failure of the main \gls{GNSS} module the backup module does not need imaging level precision to execute \gls{RTS} or \gls{RTH}.
\subsubsection{Power Supply}
\paragraph{Main Power Supply}
\paragraph{Backup Power Supply}

\subsubsection{Comparison to Commercial Options}
\paragraph{Flight Controller}
\textbf{NICE TABLE}
\paragraph{\gls{GNSS} Module}
Given the specific use case there are no other commercial options that support \gls{CAN} communication and can control the drone in case of flight controller failure. However, at just \textbf{FINAL PRICE} this is competitive with GNSS Modules without either these functionalities making it an asset to the project.

\begin{figure}[htbp]
 % Row 2: Legend + GNSS Gerber and Render
  \centering
  \begin{subfigure}[b]{0.48\textwidth}
    \includegraphics[width=\textwidth]{figs/Thomas/Custom Hardware/GPS render.png}
    \caption{GNSS Module Render}
    \label{fig:gnss_render}
  \end{subfigure}
  \hfill
  \begin{subfigure}[b]{0.48\textwidth}
    \includegraphics[width=\textwidth]{figs/Thomas/Custom Hardware/GPS gerber.png}
    \caption{GNSS Module Gerber}
    \label{fig:gnss_gerber}
  \end{subfigure}

  
  \begin{subfigure}[b]{0.48\textwidth}
    \includegraphics[width=\textwidth]{figs/Thomas/Custom Hardware/FC gerber.png}
    \caption{Custom Flight Controller Gerber}
    \label{fig:fc_gerber}
  \end{subfigure}
  \hfill
  \begin{subfigure}[b]{0.48\textwidth}
    \includegraphics[width=\textwidth]{figs/Thomas/Custom Hardware/FC render.png}
    \caption{Custom Flight Controller Render}
    \label{fig:fc_render}
  \end{subfigure}

  \vspace{1em} % space between rows

 
  
  \caption{Custom Hardware Boards: Flight Controller and GNSS Module with Legend}
  \label{fig:custom_hardware_overview}
\end{figure}

