\subsection{Design Considerations}

\subsubsection{Trace Lengths}
\paragraph{Signal Integrity and Timing}
All traces have propagation delay that increases linearly with length. This means that the longer the trace, the longer the delay. Furthermore, the longer the length, the greater effect impurities and crosstalk have creating a nosier signal. Therefore, traces should be kept as short as possible. This means that all connectors are placed at the edges and the \gls{MCU} is placed centrally.
\paragraph{Length Matching}
In synchronous communication methods the clock signal must be in sync with the data transmission. This means that the lines should be as similar in length as possible so that they arrive at the same delay. The faster the communication rate the stricter this must become as the clock signal frequency is increased. Therefore, signal speeds should be as low as possible to increase robustness. .
\paragraph{Effect of vias}
Vias allow for the transference of signals between planes which is often necessary for routing. However, they should be avoided as they introduce extra impedance, noise and delay. Furthermore, for synchronous high speed signals they should be included in trace length considerations as the line with fewer vias should be tuned to have a longer length to achieve the same delay..

\subsubsection{Protection}
\paragraph{\gls{ESD}}
\gls{ESD} occurs when there is a significant static charge built up in a ground operator or contacting surface that is then shorted in contact with the board. As electro-static voltages can be \cite{REF} this can cause significant damage to devices.
\paragraph{Touch Protection Devices}
In order to protect against \gls{ESD} the circuit must provide a low resistance pathway to the ground. This can be using \gls{TVS} diodes that above certain voltages act like a wire that diverts the current flow. They can also be mitigated using voltage dependent resistors that increase resistance with voltage and absorb the errors.  In my designs I make use of \gls{TVS} diodes due to their price and simplicity. 
\paragraph{Over current protection}
Fuses and resistors are the best way to protect a device against short circuits. Series resistors mitigate the peak current experienced by that wire. For serial debugging I ensured that all connections direct to the \gls{MCU} have series resistors in case of accidental short circuits by the user. Fuses are the best way to prevent damage from external short circuits from the power supply. This is because, in power lines series resistors dissipate too much power to be efficient. 5A fuses are used in the power supplies of both modules. 
\paragraph{Placement}
Mitigations should be as close to the possible source as possible so the least number of components and wires get damaged in failure. Therefore, all \gls{TVS} diodes are placed near touched areas or interfaces, series resistors are placed as close as possible to prevent wire transients and fuses are placed before the power is connected to the power and ground planes.

\subsubsection{Component Placement}
\paragraph{Utility Regions}
Components with similar functionalities should be restricted to specific regions, this is because it reduces trace length, prevents interference between high and low frequency signals and makes thermal management easier. These regions for the custom designed hardware can be seen in\ref{fig:custom_hardware_overview}. 
\paragraph{Thermal Considerations}
Heat dissipating elements, typically diodes and resistors, can cause damage to electronic components if not sufficiently managed. Therefore, heat sinks or controlled airflows are required if there are high power draws. This consideration is why on my devices Buck Converters of above 90\% are used instead of the less efficient \gls{LDO} which would have an efficiency of 66\% when converting from 5V to 3.3V. This, in addition to the low power draws of all other components, means that no explicit thermal management is needed.

\subsubsection{Trace Widths and Spacing}
\paragraph{Impedance Control and Thermal Management}
The impedance of a trace in inverse proportion to its cross sectional area as it acts like a wire \cite{a source}. Since the trace thickness standard as one ounce per square foot \cite{source} it is the cheapest to manufacture with and is used. Therefore, to control the impedance of a trace the width must be calibrated. \textbf{a nice table showing the widths and current capacities and specific operations}
\paragraph{Crosstalk}
When traces are too close too each other they can induce signals in each other. This means that traces should be spaced appropriately. However, since all the signals are digital or physically isolated on the board this is not a major concern but I did ensure that all traces are at least 1.5 trace widths separated as a form of best practice. Furthermore, the Radio Frequency input for the \gls{GNSS module} is kept away from all other components and the ground and voltage planes. As shown in \ref{fig:gnss_module} where the the Radio frequency line (circled in orange) is an a region not covered by the copper plane(marked as the lighter green).

\subsubsection{Layers}
\paragraph{Two Layers}
The simplest and most cost effective option in \gls{PCB} design is two copper layers separated by a dielectric. By using a copper filled regions you can create a ground plane and a power plane that effectively distribute charge and maintain voltage integrity. The simplicity of the \gls{GNSS} module means that this is the best and most cost effective option. This comes at the cost of more difficult component placement and a slightly less stable power and ground plane. However, this instability can be mitigated using distributed ceramic capacitors between the planes that filter high frequency noise in voltage levels. These capacitors are best placed next to sensors to ensure stable, low noise readings and by \gls{MCU}s so that they can maintain function.
\paragraph{Four or More Layers}
When circuits become more complex, a purpose ground and power plane in between the top and bottom plane can allow for greater packing density of components. They also make the planes more stable as all the path impedances are very low. This means that for complex designs with sensors, like the flight controller module, four layer \gls{PCB}s are the best option. Furthermore, if the complexity of the design is greater you can add further planes for analogue signals, high frequency signals or for regular signals to support greater component density.

\paragraph{Built for debugging}
Not only are there easy to use male pin headers in both designs shown in \ref{fig:GNSS_Module} \ref{fig:flight_controller} 