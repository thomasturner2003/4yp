\subsection{Unified Methodology} \label{compvis_methods}

    \subsubsection{System Architecture}
    
        \begin{enumerate}
        
            \item \textbf{Mission Planning} The user operator will select a region that they wish to map, by selecting the perimeter of the region of interest (see SECTION XX). The base-station computer will then decide on the optimal path a drone should take to get there, and cover the region of interest with the thermal sensor. A human operator then manually loads these flight instructions from the computer onto the thermal drone, after some pre-flight checks (Section \ref{missionplanning}).
            
            \item \textbf{Thermal Data Acquisition} The thermal drone autonomously captures images at 1~Hz along the planned trajectory. Post-mission, images are transferred to the base station and processed by a YOLOv11 CNN, see Section \ref{compvis_thermalcv} for details. This generates a thermal probability map \(\mathcal{P}_{\text{thermal}}(x,y)\), where each coordinate \((x,y)\) is assigned a confidence score (between 0 and 1) for landmine presence. Sparse map assembly leverages position and pose metadata to project detections onto a geographic grid  (Section \ref{orthomosaic}), avoiding computationally intensive image mosaicking.
            
            \item \textbf{Radar Data Acquisition} Regions with \(\mathcal{P}_{\text{thermal}} > 0.5\) are flagged as \textit{Points of Interest} (POIs). A radar drone autonomously surveys these POIs, collecting GPSAR Bscans along linear transects (Section \ref{points_of_interest}). A second YOLOv11 CNN processes these Bscans to produce a radar probability map \(\mathcal{P}_{\text{radar}}(x,y)\).
            
            \item \textbf{Fusion} These two probability maps, from different sensors, but over the same geographical space, are passed into a fusion algorithm , along with some environmental contextual data, returning a third, fused, probability map $\mathcal{P}_{\text{fused}}$, that is the key output of the project Section \ref{fusion}. This map can be visualised in a human-computer interface to provide actionable intelligence for demining teams
        \end{enumerate}
    
    \subsection{Layered Detection Justification and Consequences} 
    \label{sec:layered_approach}
    
        The hierarchical scanning strategy---thermal sensor for broad-area detection followed by radar for targeted verification---is fundamentally driven by the need to balance detection accuracy with operational efficiency. This approach exploits the complementary strengths of each sensor: the thermal sensor provides high recall (\(R_T\)) for rapid identification of potential mine locations, whereas the radar sensor delivers high precision (\(P_R\)) for definitive confirmation in flagged regions.
    
    \subsubsection{Information Entropy Reduction} 
    
        The thermal sensor acts as an entropy-reducing filter. Let \(\rho_0 = P(\text{Mine})\) denote the initial mine density (prior probability), \(R_T = P(\text{Thermal Flag} \mid \text{Mine})\) the thermal recall (true positive rate), and \(F_T = P(\text{Thermal Flag} \mid \text{No Mine})\) the thermal false positive rate. After thermal scanning, the posterior probability \(\rho_1\) of a mine in a flagged region (\(\mathcal{T}^+\)) is derived from Bayes' theorem:
        
        \begin{equation}
            \rho_1 = P(\text{Mine} \mid \mathcal{T}^+) = \frac{P(\mathcal{T}^+ \mid \text{Mine}) P(\text{Mine})}{P(\mathcal{T}^+)} = \frac{R_T \rho_0}{R_T \rho_0 + F_T (1 - \rho_0)} = \frac{\rho_0}{\rho_0 + \frac{F_T}{R_T} (1 - \rho_0)}
        \end{equation}
        
        Under the reasonable assumption that \(\frac{F_T}{R_T}\ll 1\), this posterior probability satisfies \(\rho_1 \gg \rho_0\), indicating that flagged regions exhibit higher mine density. The radar subsequently refines \(\rho_1\) to a fused precision \(P_{\text{fused}} > \rho_1\), as detailed in Section~\ref{sec:fusion}.
    
    %\subsubsection{Time Efficiency Analysis} 
    %Let \(A\) represent the total survey area, \(t_T\) the time per unit area for thermal scanning, and \(t_R\) the time per unit area for radar scanning. The total time for layered scanning comprises two components: thermal coverage of the entire area and radar scrutiny of high-probability regions:
    
    %\begin{equation}
        %t_{\text{layered}} = \underbrace{t_TA}_{\text{Thermal}} + \underbrace{t_R A (R_T \rho_0 + F_T (1 - \rho_0))}_{\text{Radar on } \mathcal{T}^+}.
    %\end{equation}
    
    %For representative parameters (\(R_T = 0.95\), \(F_T = 0.05\), \(\rho_0 = 0.05\), \(t_T = 0.1t_R\)):
    
    %\begin{equation}
        %t_{\text{layered}} = 0.1t_R A + t_R A \left((0.95 \times 0.05) + (0.05 \times 0.95)\right) =  0.195t_R A.
    %\end{equation}
    
    %A simultaneous full-coverage deployment, requiring both sensors to scan the entire area independently, would demand:
    
    %\begin{equation}
        %t_{\text{parallel}} = \max(t_T, t_R) %\cdot A = t_R A,
    %\end{equation}
    
    %as the radar's slower scan rate (\(t_R > t_T\)) dictates operational tempo. The layered approach therefore achieves an \textbf{80.5\% reduction} in operational time compared to exhaustive parallel scanning with these representative parameters.
    
    %Continuing this example, $\rho_1 \approx \text{50\%}$, meaning that the density of mines in the flagged regions was ten times that of the original area. This underscores the reasoning behind only using the radar sensors in the flagged, mine rich regions.
    
    \subsubsection{Conclusion} 
    
        The hierarchical sensor integration described ensures much faster detection, with minimal compromise on accuracy, by leveraging the high recall, high speed thermal sensors, and the slower, high precision radar sensor a very efficient manner.
        
        The subsequent section, Section~\ref{compvis_implementation}, elaborates on sensor-specific implementations, where we will quantify the precision and recall of the thermal and radar systems.